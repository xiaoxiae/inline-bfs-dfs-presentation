% notes to myself:
% - starred version (e.g. \section*{Section name}) disables the (sub)section page.

\DeclareMathSizes{20}{20}{20}{20}

\documentclass{beamer}

\usetheme[nofirafonts]{focus}

% shenanigans to allow UTF-8
\usepackage{fontspec}

\setsansfont{Fira Sans Book}[%
	ItalicFont = {Fira Sans Book Italic},
	BoldFont = {Fira Sans SemiBold},
	BoldItalicFont = {Fira Sans SemiBold Italic}]

\usefonttheme[onlymath]{serif}

% shamelessly stolen from https://www.patrickbaylis.com/posts/2018-10-11-beamer-resizing/
\usepackage{adjustbox}
\makeatletter
\newcommand{\fitimage}[2][\@nil]{
	\begin{figure}
		\begin{adjustbox}{width=0.9\textwidth, totalheight=\textheight-2\baselineskip-2\baselineskip,keepaspectratio}
			\includegraphics{#2}
		\end{adjustbox}
		\def\tmp{#1}%
	 \ifx\tmp\@nnil
			\else
			\caption{#1}
		\fi
	\end{figure}
}
\makeatother

% translations
\renewcommand{\figurename}{Diagram}
\renewcommand*{\proofname}{Důkaz}

% bullets
\setbeamertemplate{itemize item}{$\bullet$}
\setbeamertemplate{itemize subitem}{$\bullet$}

% code
\usepackage{minted}
\setminted[python]{
	linenos,
	mathescape=true,
	escapeinside=||,
	autogobble,
	obeytabs=true,
	tabsize=4}

% ---


\title{In-place BFS a DFS}
\subtitle{v lineárním čase na modelu RAM}
\date{Tomáš Sláma \hfill 21. 11. 1999} % please don't judge me

\begin{document}
	\begin{frame}
		\maketitle
	\end{frame}
	
	\section{Úvod}
	\begin{frame}{Předpoklady}
		\begin{itemize}
			\item pracujeme s modelem RAM
			\begin{itemize}
				\item paměť je pole slov velikosti $w$
				\item operace se slovy (čtení, psaní, přístupy) jsou konstantní
				\item vstup je na prvních $N \in \mathbb{N}$ slovech
				\item velikost $w = \Omega\left(\log N\right)$
			\end{itemize}
			\vfill
			\item vstup \textit{může být měněn} v průběhu výpočtu, ale na jeho konci musí být v původním stavu
			\vfill
			\item graf je zadán v setříděném seznamu sousedů
		\end{itemize}

		\fitimage[setřízená reprezentace]{images/sorted.png}
	\end{frame}

	\section{DFS}
	\begin{frame}{Přehled}
		This is a simple frame.
	\end{frame}

	\subsection{Reprezentace}
	\begin{frame}{setřízená $\rightarrow$ pointerová}
		\begin{lemma}
			Setříděná reprezentace jde do pointerové reprezentace převést in-place v čase $\mathcal{O}\left(n\right)$.
		\end{lemma}

		\begin{proof}
			Nastavíme $A[i] = T\left[A\left[i\right]\right] \forall i \in \left\{n + 2, \ldots, n + m + 2\right\}$ \\
			a $T\left[v\right] = v$ (pro vrcholy $v$ stupně $0$).
		\end{proof}
	\end{frame}

	\begin{frame}{pointerová $\longleftrightarrow$ prohozená}
		\begin{lemma}
			Pointerová jde do prohozené reprezentace (a zpět) převést in-place v čase $\mathcal{O}\left(n\right)$.
		\end{lemma}

		\begin{proof}
			Převod $\rightarrow$: $$T[i] = A[T[i]]\ \text{a}\ A[T[i]] = i\quad \forall i \in \left\{1, \ldots, n\right\}, i \neq T[v]$$
			Převod $\leftarrow$: $$T[A[i]] = i\ \text{a}\ A[i] = T[A[i]]\quad \forall i \in \left\{n + 2, \ldots, n + m + 2\right\}, A[i] < n$$
		\end{proof}
	\end{frame}

	\begin{frame}{prohozená $\rightarrow$ setřízená}
		\begin{lemma}
			Prohozená jde do setřízené reprezentace převést in-place v čase $\mathcal{O}\left(n\right)$.
		\end{lemma}

		\begin{proof}
			TODONejprve nastavíme $A[i] = A[A[i]]\ \forall i, A[i] > n, i !=$.
			\vfill
			Poté procházíme $A$ a prohazujeme ještě neprohozené vrcholy $v \in \left\{1, \ldots, n\right\}$ s pozicemi $p$ a nastavujeme $T[v] = p\ \text{a}\ A[p] = T[v]$
			Poté opravíme vrcholy stupně $0$ tím, že 
			$$T[i] = T[i - 1]\ \forall i \in \left\{1, \ldots, n\right\}, T[i] \le n$$

		\end{proof}
	\end{frame}

	\subsection{Přehled}
	\begin{frame}{setřízená $\rightarrow$ pointerová}
		\begin{lemma}
			Setříděná reprezentace jde do pointerové reprezentace převést in-place v čase $\mathcal{O}\left(n\right)$.
		\end{lemma}

		\begin{proof}
			Nastavíme $A[i] = T\left[A\left[i\right]\right] \forall i \in \left\{n + 2, \ldots, n + m + 2\right\}$ a také $T\left[v\right] = v$ pro vrcholy $v$ stupně $0$.
		\end{proof}
	\end{frame}

	\section{BFS}
	\begin{frame}{Přehled}
		\begin{itemize}
			\item využijeme setřízenosti $T$ a zkomprimujeme jej na $\mathcal{T}$
			\begin{itemize}
				\item uvolníme tím lineární počet bitů
				\item zachováme konstantní přístup
			\end{itemize}
			\item ve volném místě vytvoříme color choice dictionary, ve kterém budeme ukládat barvy vrcholů
			\item $\mathcal{T}$ dekomprimujeme na $T$
		\end{itemize}
	\end{frame}

	\begin{frame}{Komprimace $T$}
		\begin{itemize}
			\item potřebujeme $nc$ bitů na uložení CCD~\cite{DBLP:journals/corr/abs-1809-07661}
				\vfill
			\item najdeme pozice, kde se mění $c + 1$ MSb ve slovech z $T$, budeme je považovat za nepoužitá
			\begin{itemize}
				\item díky setřízenosti jich bude právě $2^{c - 1}$
			\end{itemize}
			\begin{block}{počet volných bitů}
				\centering
				$nw - n(w - (c + 1)) = n(c + 1) = nc + n$
			\end{block}
			\item omezení: $n \ge 2^{c + 1}w$, abychom mohli uložit pozice a $c + 1$
		\end{itemize}

		\fitimage[komprimovaná reprezentace $\mathcal{T}$]{images/compression.png}
	\end{frame}

	\begin{frame}{Získávání bitů z $\mathcal{T}$}
		...
	\end{frame}


	\begin{frame}{Color choice dictionary}
		\begin{itemize}
			\item fdsa
			\vfill
			\item operace v konstantním čase~\cite{DBLP:journals/corr/abs-1809-07661}:
			\begin{itemize}
				\item \texttt{setColor(v, c)} -- nastaví barvu $v$ na $c$
				\item \texttt{getColor(v)} -- získá barvu $v$
				\item \texttt{choice(c)} -- získá libovolný $v$ s barvu $c$
			\end{itemize}
		\end{itemize}
	\end{frame}

	\begin{frame}[fragile]{Průběh BFS}
		\small
		\begin{minted}{python}
			D = ColorChoiceDicrionary(WHITE, LIGHT, DARK, BLACK)
			D.setColor(start, LIGHT)

			while choice(LIGHT) is not None:
				while choice(LIGHT) is not None:
					v = D.choice(LIGHT)

					# open all white neighbours
					for i in range(|$\mathcal{T}[v]$|, |$\mathcal{T}[v + 1]$|):
						if D.getColor(|$A[i]$|) is WHITE:
							D.setColor(|$A[i]$|, DARK)

					D.setColor(v, BLACK)  # close the node

				LIGHT, DARK = DARK, LIGHT  # next round
		\end{minted}
	\end{frame}

	\begin{frame}[focus]
		Díky za pozornost!
	\end{frame}
	
	\appendix
	\begin{frame}{Zdroje}
		\nocite{*}
		\bibliography{presentation}
		\bibliographystyle{plain}
	\end{frame}
\end{document}
